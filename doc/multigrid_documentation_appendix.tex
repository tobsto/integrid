\appendix
\chapter{Loggrid parameter}
\section{Boundary conditions}
\label{sec:app_loggrid_boundary}
In the following we will show how the four boundary conditions (\ref{eqn:loggrid_boundary_conditions})
\begin{align}
	\omega(0)&=\omega_{min}   \label{eqn:loggrid_boundary_condition_1}\\
	\omega(N_l-1)&=\omega_k-\omega_0  \label{eqn:loggrid_boundary_condition_2}\\
	\omega(N_l+N_k)&=\omega_k+\omega_0   \label{eqn:loggrid_boundary_condition_3}\\
	\omega(N_l+N_k+N_r)&=\omega_{max}   \label{eqn:loggrid_boundary_condition_4}
\end{align}
determine the four parameters $i_1$, $i_2$, $c_1$ and $c_2$ in the definition of the logarithmic grid~(\ref{eqn:loggrid_definition}):
\begin{equation*}
	\omega(i)=\begin{cases}
		-\exp(-c_1(i-i_1)) + \omega_k 		\quad & i\in\{0,\dots,N_l-1\} \\ 
		\omega_k-\omega_0+d\omega_k(i-N_l)	\quad & i\in\{N_l,\dots,N_l+N_k-1\} \\ 
		\exp(c_2(i-i_2-N_l-N_k)) + \omega_k 	\quad & i\in\{N_l+N_k,\dots,N_l+N_k+N_r\}
	\end{cases}
\end{equation*}

\vspace{1cm}
\noindent\underline{Determine $i_1$ and $c_1$:}\\

\noindent Insert \ref{eqn:loggrid_definition} into (\ref{eqn:loggrid_boundary_condition_1}) and (\ref{eqn:loggrid_boundary_condition_2}):
\begin{align}
	-\exp(-c_1(0-i_1))+\omega_k&=\omega_{min} \quad \Leftrightarrow \quad \log(\omega_k-\omega_{min})=c_1 i_1 \label{eqn:app_loggrid_boundary_condition_1_1}\\
	-\exp(-c_1(N_l-1-i_1))+\omega_k&=\omega_k-\omega_0 \quad \Leftrightarrow \quad \log(\omega_0)=-c_1(N_l-1-i_1) \label{eqn:app_loggrid_boundary_condition_2_1}
\end{align}
Difference (\ref{eqn:app_loggrid_boundary_condition_1_1}) $-$ (\ref{eqn:app_loggrid_boundary_condition_2_1}):
\[
	\log\left(\frac{\omega_k-\omega_{min}}{\omega_0}\right)=c_1 (N_l-1) \quad \boxed{\Rightarrow c_1=\frac{\log\left(\frac{\omega_k-\omega_{min}}{\omega_0}\right)}{N_l-1}}
\]
Insert into (\ref{eqn:app_loggrid_boundary_condition_2_1}):
\[
	\boxed{i_1=\frac{\log(\omega_k-\omega_{min})}{c_1}}
\]

\newpage
\vspace{1cm}
\noindent\underline{Determine $i_2$ and $c_2$:}\\

\noindent Insert \ref{eqn:loggrid_definition} into (\ref{eqn:loggrid_boundary_condition_1}) and (\ref{eqn:loggrid_boundary_condition_2}):
\begin{align}
	\exp(c_2(0-i_2))+\omega_k&=\omega_k+\omega_0 \quad \Leftrightarrow \quad \log(\omega_0)=c_2 (0-i_2) \label{eqn:app_loggrid_boundary_condition_3_1}\\
	\exp(c_2(N_r-i_2))+\omega_k&=\omega_{\max} \quad \Leftrightarrow \quad \log(\omega_{max}-\omega_k)=c_2(N_r-i_2) \label{eqn:app_loggrid_boundary_condition_4_1}
\end{align}
Difference (\ref{eqn:app_loggrid_boundary_condition_4_1}) $-$ (\ref{eqn:app_loggrid_boundary_condition_3_1}):
\[
	\log\left(\frac{\omega_{max}-\omega_k}{\omega_0}\right)=c_2 (N_r)\\\Rightarrow \boxed{c_2=\frac{\log\left(\frac{\omega_{max}-\omega_k}{\omega_0}\right)}{N_r}}
\]
Insert into (\ref{eqn:app_loggrid_boundary_condition_4_1}):
\[
	\boxed{i_2=-\frac{\log(\omega_{max}-\omega_k)}{c_2} + N_r}
\]
\section{Maximal resolution}
\label{sec:app_loggrid_max_resolution}
The maximal resolution of the exponential grid regions to the left (region I), $d\omega_{min}^{l}$ and to the right (region III), $d\omega_{min}^{r}$ of the center point in a logarithmic grid reads
\begin{align*}
	d\omega_{min}^{l} & = \omega(N_l-1)-\omega(N_l-2) \notag \\
	& = -\exp(-c_1(N_l-1-i_1))+exp(-c_1(N_l-2-1_1)) \notag \\
	& = \exp(-c_1(N_l-1-i_1))(e^{c_1}-1)
\end{align*}
and
\begin{align*}
	d\omega_{min}^{r} & = \omega(N_l+N_k+1)-\omega(N_l+N_k) \notag \\
	& = \exp(-c_2 i_2)(e^{c_2}-1)
\end{align*}
The resolution of the linear grid (region II) will be the minimum of both
\[
	d\omega_k = min( d\omega_{min}^{l} ,\,d\omega_{min}^{r})
\]
\chapter{Cutting grid regions}\label{chapter:app_cutting}
\index{Cutting points}
\section{Calculation of cutting points}\label{sec:app_cutting_points}
In this section we will calculate the points of equal grid point density $\omega_s$ between two grids of equidistant, tangential and logarithmic type. We will use the corresponding equations for the grid point density of these grid types, i.e.~equations \ref{eqn:equigrid_grid_point_density}, \ref{eqn:tangrid_grid_point_density} and \ref{eqn:loggrid_grid_point_density} in order to solve equation \ref{eqn:cutting_point}. We will go through all possible combinations for the left and right grid region, where we denote the left grid region parameters by a superscript $l$ and the right grid region parameter by a superscript $r$.
\begin{enumerate}
	\item {\bf loggrid/loggrid}:
		\begin{align*}
			\frac{1}{c_2^l(\omega_s-\omega_k^l)}\stackrel{!}{=}\frac{1}{c_1^r(\omega_k^r-\omega_s)}
		\end{align*}
		\[
		 	\Rightarrow \boxed{\omega_s=\frac{c_2^l\omega_k^l + c_1^r\omega_k^r}{c_2^l+c_1^r}}
		\]

	\item {\bf loggrid/tangrid}:
		\begin{align*}
			&\frac{1}{c_2^l(\omega_s-\omega_k^l)}\stackrel{!}{=}\frac{1}{c^r\Delta u^r\left(1+\left(\frac{\omega_s-\omega_c^r}{c^r}\right)^2\right)}\\
			&\Leftrightarrow c_2^l(\omega_s-\omega_k^l) = c^r \Delta u^r + \frac{\Delta u^r}{c^r} (\omega_s-\omega_c^r)^2\\
			&\Leftrightarrow \omega_s^2 - \left(2\omega_c^r + \frac{c^r c_2^l}{\Delta u^r}\right) \omega_s + \left((\omega_c^{r})^2 + (c^r)^2 + \frac{c^r c_2^l}{\Delta u^r} \omega_k^l \right) =0
		\end{align*}
		\[
		 	\Rightarrow \boxed{\omega_s= \omega_c^r + \frac{c^r c_2^l}{2 \Delta u^r} - \sqrt{\underbrace{\left(\omega_c^r + \frac{c^r c_2^l}{2 \Delta u^r}\right)^2 - (\omega_c^{r})^2 - (c^r)^2 - \frac{c^r c_2^l}{\Delta u^r} \omega_k^l }_{<0 \text{ if no solution to \ref{eqn:cutting_point} exists}}}}
		\]
	\item {\bf tangrid/loggrid}:
		\begin{align*}
			&\frac{1}{c^l\Delta u^l\left(1+\left(\frac{\omega_s-\omega_c^l}{c^l}\right)^2\right)}\stackrel{!}{=}\frac{1}{c_1^r(\omega_k^r-\omega_s)}
		\end{align*}
		\[
		 	\Rightarrow \boxed{\omega_s= \omega_c^r - \frac{c^l c_1^r}{2 \Delta u^l} + \sqrt{\underbrace{\left(\omega_c^l + \frac{c^l c_1^r}{2 \Delta u^l}\right)^2 - (\omega_c^{l})^2 - (c^l)^2 + \frac{c^l c_1^r}{\Delta u^l} \omega_k^r }_{<0 \text{ if no solution to \ref{eqn:cutting_point} exists}} }}
		\]
	\item {\bf tangrid/tangrid}:
		\begin{align*}
			&\frac{1}{c^l\Delta u^l\left(1+\left(\frac{\omega_s-\omega_c^l}{c^l}\right)^2\right)}\stackrel{!}{=}\frac{1}{c^r\Delta u^r\left(1+\left(\frac{\omega_s-\omega_c^r}{c^r}\right)^2\right)}\\
			&\Leftrightarrow c^l\Delta u^l + \frac{\Delta u^l}{c^l}(\omega_s - \omega_c^l)^2 - c^r \Delta u^r - \frac{\Delta u^r}{c^r}(\omega_s - \omega_c^r)^2 = 0 \\
			&\Leftrightarrow \left( \frac{\Delta u^l}{c^l} - \frac{\Delta u^r}{c^r}\right) \omega_s^2 - \left( \frac{2\Delta u^l \omega_c^l}{c^l} - \frac{2 \Delta u^r \omega_c^r}{c^r} \right) \omega_s + \frac{\Delta u^l}{c^l}(\omega_c^l)^2 - \frac{\Delta u^r}{c^r}(\omega_c^r)^2 + c^l\Delta u^l c^r \Delta u^r = 0\\
			&\Leftrightarrow \omega_s^2 - \underbrace{\biggl(\frac{2\Delta u^l \omega_c^l c^r}{\Delta u^l c^r - \Delta u^r c^l} - \frac{2\Delta u^r \omega_c^r c^l}{\Delta u^l c^r - \Delta u^r c^l} \biggr)}_{:=p} \omega_s + \underbrace{\biggl(\frac{c^r \Delta u^l (\omega_c^l)^2 - c^l \Delta u^r (\omega_c^r)^2 + \Delta u^l c^r (c^l)^2 - \Delta u^r c^l (c^r)^2}{\Delta u^l c^r - \Delta u^r c^l}\biggr)}_{:=q}=0
		\end{align*}
		\[
			\Rightarrow \boxed{\omega_s= \frac{p}{2} \pm \sqrt{\underbrace{\frac{p^2}{4}-q}_{\stackrel{<0 \text{ if no solution}}{\text{to \ref{eqn:cutting_point} exists} } } }}
		\]
		The plus or minus sign is chosen such, that $\omega_s$ lies inside $[\omega_c^l, \omega_c^r]$.

	\item {\bf loggrid/equigrid}:
		\begin{align*}
			\frac{1}{c_2^l(\omega_s-\omega_k^l)}\stackrel{!}{=}\frac{1}{d\omega^r}
		\end{align*}
		\[
			\Rightarrow \boxed{\omega_s= \omega_k^l + \frac{d\omega^r}{c_2^l}}
		\]
	\item {\bf equigrid/loggrid}:
		\begin{align*}
			\frac{1}{d\omega^l}\stackrel{!}{=}\frac{1}{c_1^r(\omega_k^r-\omega_s)}
		\end{align*}
		\[
			\Rightarrow \boxed{\omega_s= \omega_k^r - \frac{d\omega^l}{c_1^r}}
		\]
	\item {\bf tangrid/equigrid}:
		\begin{align*}
			\frac{1}{c^l\Delta u^l\left(1+\left(\frac{\omega_s-\omega_c^l}{c^l}\right)^2\right)}\stackrel{!}{=}\frac{1}{d\omega^r}
		\end{align*}
		\[
			\Rightarrow \boxed{\omega_s= \omega_c^l + c^l \sqrt{\underbrace{\frac{d\omega^r}{c^l\Delta u^l} -1}_{\stackrel{<0 \text{ if no solution}}{\text{to \ref{eqn:cutting_point} exists} } }}}
		\]
	\item {\bf equigrid/tangrid}:
		\begin{align*}
			\frac{1}{d\omega^l}\stackrel{!}{=}\frac{1}{c^r\Delta u^r\left(1+\left(\frac{\omega_s-\omega_c^r}{c^r}\right)^2\right)}
		\end{align*}
		\[
			\Rightarrow \boxed{\omega_s= \omega_c^r - c^r \sqrt{\underbrace{\frac{d\omega^l}{c^r\Delta u^r} -1}_{\stackrel{<0 \text{ if no solution}}{\text{to \ref{eqn:cutting_point} exists} } }}}
		\]
	\item {\bf equigrid/equigrid}:
		\begin{align*}
			\frac{1}{d\omega^l}\stackrel{!}{=}\frac{1}{d\omega^r}
		\end{align*}
		\[
			\Rightarrow 
			\boxed{\omega_s= \begin{cases}
							\omega_l^r \quad \text{ if } d\omega^r \leq d\omega^l \\
							\omega_r^l \quad \text{ if } d\omega^r > d\omega^l
			                             \end{cases}
			}
		\]
\end{enumerate}


\section{Adjust grid region parameters}\label{sec:app_cutting_gr}
After a cutting point according to section \ref{sec:cutting_procedure} is found, the left or right part of a grid region will be cut. The cutting point $\omega_s$ must be in the interval $[\omega_l, \omega_-]$ for cutting the left part of a grid region and in the interval $[\omega_+, \omega_r]$ for cutting the right part of a grid region. In the following we will show how the corresponding grid region parameter must be altered in order to preserve the maximal grid resolution. 
\begin{enumerate}
	\item {\bf loggrid}, cut left
	\begin{align*}
		N_l&\to\frac{1}{c_1} \log\left(\frac{\omega_k -\omega_s}{\omega_0}\right) \\
		\omega_l&\to\omega_s
	\end{align*}
	\item {\bf loggrid}, cut right
	\begin{align*}
		N_r&\to\frac{1}{c_2} \log\left(\frac{\omega_s -\omega_k}{\omega_0}\right) \\
		\omega_r&\to\omega_s
	\end{align*}
	\item {\bf tangrid}, cut left
	\begin{align*}
                 M&\to\frac{1}{\Delta u} \arctan \left( \frac{\omega_r-\omega_c}{c} \right)-\arctan \left( \frac{\omega_s-\omega_c}{c} \right)\\
		\omega_l&\to\omega_s
	\end{align*}
	\item {\bf tangrid}, cut right
	\begin{align*}
                 M&\to\frac{1}{\Delta u} \arctan \left( \frac{\omega_s-\omega_c}{c} \right)-\arctan \left( \frac{\omega_l-\omega_c}{c} \right)\\
		\omega_r&\to\omega_s
	\end{align*}
	\item {\bf equigrid}, cut left
	\begin{align*}
                 M&\to\frac{\omega_r-\omega_s}{\omega_r-\omega_l} M\\
		\omega_l&\to\omega_s
	\end{align*}
	\item {\bf equigrid}, cut right
	\begin{align*}
                 M&\to\frac{\omega_s-\omega_l}{\omega_r-\omega_l} M\\
		\omega_l&\to\omega_s
	\end{align*}
\end{enumerate}
Note, that all the grid point numbers $N_l$, $N_r$, $M$, $\dots$ are chosen to be at least equal to $3$ for numerical reasons. This is in accordance with the choice of the buffer regions (see \ref{tab:grid_region_types}).









