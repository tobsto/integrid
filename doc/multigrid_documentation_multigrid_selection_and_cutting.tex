\chapter{Selection and Cutting in the Multigrid} \label{chapter:multigrid_selection_and_cutting}
In this chapter we the details of the selection and cutting procedure will be explained. Moreover, we will show how to keep track of inverse mapping while adding fundamental and special grid regions to the multigrid. For 
\section{Cutting procedure}\label{sec:cutting_procedure}
\index{Cutting procedure}

\section{Subgrid and inserting grid regions}\label{sec:subgrids_and_insert}
\index{Subgrid}
\index{Inverse of the multigrid}
If all fundamental grid regions have gone through the cutting procedure explained in \ref{sec:cutting_procedure}, we are left with a bunch of non-intersection grid regions. These grid regions are now ready to be inserted into the basic grid region. But there is one difficulty we have not mentioned so far which is the fact that one has to take care of the inverse. Since all grid regions consist of one of the simple grid classes from chapter \ref{chapter:simple_grids}, their inverse mapping is know.

\section{Inverse of the multigrid}\label{sec:inverse_of_the_multigrid}
\index{Inverse of the multigrid}



