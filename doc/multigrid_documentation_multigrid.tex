\chapter{Multigrid} \label{chapter:multigrid}
\index{Multigrid}
\section{Overview}
A multigrid consist of a single basic grid which defines the outer boundaries of the multigrid and multiple grid regions (GR). Each of these grid regions can be equidistant, tangential or logarithmic. If the grid regions overlap each other, they will be cut in such a way, that the grid resolution remains constant while crossing over the cutting point. It could also happen that a particular grid region is forced out by another grid region because the purpose of the former one is already served by the latter one. The first added grid region defines the outer boundaries of the multigrid and it is excluded from the cutting procedure. It is called basic grid region (BGR).

A multigrid is created in the following way (see listing \ref{lst:create_multigrid_sketch}). First it has to be initialized. Afterwards one can add, replace and remove multiple grid regions on the multigrid. This will be explained in detail in section \ref{sec:grid_regions}. Finally the member function \texttt{create} in invoked. Hereby the cutting and selection procedure will be performed and the multigrid will be created. 
\begin{lstlisting}[caption={Creating a multigrid},	label={lst:create_multigrid_sketch}]
multigrid mgrid;
... 
... //add, replace and remove grid regions
...
mgrid.create()
\end{lstlisting}
The multigrid is also derived from the grid class and can therefore be used exactly in the same way then the other grids (see chapter \ref{sec:using_the_grid_classes})

\section{Grid regions}\label{sec:grid_regions}
\index{Grid region}
In this chapter we will introduce the notion of a grid region. We will assume, that the purpose of a grid region is to resolve some feature of the integrand function which is located a a single point. This will be the center point of the grid region $\omega_c$. Together with the left and right boundaries $\omega_l$ and $\omega_r$, the type of grid region (equidistant, tangential or logarithmic) and the grid region parameters (e.g. $\omega_k$ and $\omega_0$ for the loggrid) these are the defining properties of a grid region. The grid region is implemented as a class named \texttt{gridregion} and is defined in the file \texttt{multigrid.h}. Its defining member variables are shown in table \ref{tab:grid_region_defining_members}.
\begin{table}[h]
	\begin{center}
		\begin{tabular}{ll}
		Name & Description \\ 
		\hline
		$\omega_c$  & Center point \\
		$\omega_l$  & Lower boundary \\
		$\omega_r$  & Upper boundary \\
		\texttt{type}  & Type of grid region \\
		\texttt{id}  & Name of the grid region (optional) \\
		 & Grid region parameters (e.g. $\omega_k$ and $\omega_0$ for the loggrid) \\
		\end{tabular}
	\end{center}
	\caption{Defining properties of the grid region}
	\label{tab:grid_region_defining_members}
\end{table}

It obvious, that a possible cutting point can not be arbitrary narrow to the center point $\omega_c$. For example in the loggrid case, for numerical reasons at least three point of the exponential grid regions must survive in order to have a starting, an intermediate and an endpoint of the grid. Also the corresponding weights should not be smaller than the machine precision. Therefore one introduces a buffer region around $\omega_c$ which is defined by its upper and lower boundaries $\omega_+$ and $\omega_-$. In section \ref{sec:cutting_procedure} we will see, that in order to calculate the cutting point according to the grid point density (grid resolution) the maximal grid resolution on both sides of the center point $d\omega_{min}^l$ and $d\omega_{min}^r$ is necessary. Hereby we assume that the grid point density is monotonically increasing from the boundaries towards the center of the grid region. (This is true for all three kinds of grids from chapter \ref{chapter:simple_grids}). Table \ref{tab:grid_region_derived_members} shows the relevant derived properties of the grid region class, i.e. they are calculated directly out of the defining properties.
\begin{table}[h]
	\begin{center}
		\begin{tabular}{ll}
		Name & Description \\ 
		\hline
		$\omega_+$  & Upper boundary of the buffer region around $\omega_c$ \\
		$\omega_-$  & Lower boundary of the buffer region around $\omega_c$ \\
		$d\omega_{min}^l$  & Maximal resolution left to $\omega_c$ \\
		$d\omega_{min}^r$  & Maximal resolution right to $\omega_c$ \\
		\end{tabular}
	\end{center}
	\caption{Relevant derived properties of the grid region}
	\label{tab:grid_region_derived_members}
\end{table}

Note that there are actually more member variables. But since they we will never appear to the user will not discuss them here. In the following we will introduce the different kinds of grid regions. There are equidistant, tangential and logarithmic grid regions. Naturally, the boundaries of a particular grid region $\omega_l$ and $\omega_r$ will correspond to the boundary $\omega_{min}$ and $\omega_{max}$ of the simple grids from chapter \ref{chapter:simple_grids}. In table \ref{tab:grid_region_types} the mapping from the grid class member variables to the grid region member variables is shown. As mentioned earlier, the boundaries of the buffer region around the center point are chosen such, that there will be 3 points left. In the case of the logarithmic grid region, the buffer region always includes the linear region II of the loggrid.
\begin{table}[h]
	\begin{center}
		\begin{tabular}{llll}
		Grid region & equidistant & tangential & logarithmic  \\ 
		variable    &             &            &              \\
		\hline 
		$\omega_c$    & -               & $\omega_c$     & $\omega_k$     \\
		$\omega_l$    & $\omega_{min}$  & $\omega_{min}$ & $\omega_{min}$ \\
		$\omega_r$    & $\omega_{max}$  & $\omega_{max}$ & $\omega_{max}$ \\
		\texttt{type} & \texttt{'equi'} & \texttt{'tan'} & \texttt{'log'} \\
		$\omega_+$         & $\omega_c+3d\omega$ & $\omega_c+3c\Delta u$ & $\omega(i=N_l+N_k+3)$ \\
		$\omega_-$         & $\omega_c-3d\omega$ & $\omega_c-3c\Delta u$ & $\omega(i=N_l-3)$  \\
		$d\omega_{min}^l$  & $d\omega$ & $c\Delta u$ & see Appendix \ref{sec:app_loggrid_max_resolution} \\
		$d\omega_{min}^r$  & $d\omega$ & $c\Delta u$ & see Appendix \ref{sec:app_loggrid_max_resolution} \\
		\end{tabular}
	\end{center}
	\caption{Relation of grid class and grid region member variables}
	\label{tab:grid_region_types}
\end{table}

\subsection{Adding grid regions}\label{subsec:grid_region_adding}
\index{Grid region!equidistant}
\index{Grid region!tangential}
\index{Grid region!logarithmic}
\index{Grid region!add}
In this section we introduce the syntax for adding grid regions to the multigrid. (See section \ref{subsec:grid_region_examples} for examples).
An equidistant grid region consists of an instance of an equigrid (see \ref{sec:equigrid}). It is added to the multigrid by the member function 
\begin{lstlisting}
void add_gr_equi
\end{lstlisting}
Its arguments are shown in table \ref{tab:add_gr_equi}.

\begin{table}[h]
	\begin{center}
		\begin{tabular}{lll}		
		Argument  & Type & Description \\ \hline
		\nth{1}   & \texttt{int}    & Number of grid points ($M$) \\ 
		\nth{2}   & \texttt{double} & Lower boundary ($\omega_l$) \\ 
		\nth{3}   & \texttt{double} & Upper boundary ($\omega_r$) \\ 
		\nth{4}   & \texttt{double} & Center point ($\omega_c$) \\ 
		(\nth{5}) & \texttt{string} & Name of the grid region (\texttt{id})(optional)\\ 
		\end{tabular}
	\end{center}
	\caption{Arguments of the member function \texttt{add\_gr\_equi}}
	\label{tab:add_gr_equi}
\end{table}

A tangential grid region consists of an instance of an tangrid (see \ref{sec:tangrid}). It is added to the multigrid by the member function 
\begin{lstlisting}
void add_gr_tan
\end{lstlisting}
Its arguments are shown in table \ref{tab:add_gr_tan}.

\begin{table}[h]
	\begin{center}
		\begin{tabular}{lll}		
		Argument  & Type & Description \\ \hline
		\nth{1}   & \texttt{int}    & Number of grid points ($M$) \\ 
		\nth{2}   & \texttt{double} & Lower boundary ($\omega_l$) \\ 
		\nth{3}   & \texttt{double} & Upper boundary ($\omega_r$) \\ 
		\nth{4}   & \texttt{double} & Center point ($\omega_c$) \\ 
		\nth{5}   & \texttt{double} & Controls sharpness ($c$) \\ 
		(\nth{6}) & \texttt{string} & Name of the grid region (\texttt{id})(optional)\\ 
		\end{tabular}
	\end{center}
	\caption{Arguments of the member function \texttt{add\_gr\_tan}}
	\label{tab:add_gr_tan}
\end{table}

A logarithmic grid region consists of an instance of an loggrid (see \ref{sec:loggrid}). It is added to the multigrid by the member function 
\begin{lstlisting}
void add_gr_log
\end{lstlisting}
Its arguments are shown in table \ref{tab:add_gr_log}.

\begin{table}[h]
	\begin{center}
		\begin{tabular}{lll}		
		Argument  & Type & Description \\ \hline
		\nth{1}   & \texttt{int}    & Number of grid points in region I ($N_l$) \\ 
		\nth{2}   & \texttt{int}    & Number of grid points in region III ($N_r$) \\ 
		\nth{3}   & \texttt{double} & Lower boundary ($\omega_l$) \\ 
		\nth{4}   & \texttt{double} & Upper boundary ($\omega_r$) \\ 
		\nth{5}   & \texttt{double} & Center point ($\omega_k$) \\ 
		\nth{6}   & \texttt{double} & Half width of region II ($\omega_0$) \\ 
		(\nth{7}) & \texttt{string} & Name of the grid region (\texttt{id})(optional)\\ 
		\end{tabular}
	\end{center}
	\caption{Arguments of the member function \texttt{add\_gr\_log}}
	\label{tab:add_gr_log}
\end{table}

\subsection{Replacing and removing grid regions}\label{subsec:grid_region_replace_remove}
\index{Grid region!equidistant}
\index{Grid region!tangential}
\index{Grid region!logarithmic}
\index{Grid region!replace}
\index{Grid region!remove}
We have seen, that if a grid region is added to the multigrid, it is optional give a name to this grid region. In contrast to that, you can only replace or remove grid regions which have a name. One can check if a grid region with the name \texttt{id} exists, by calling the function
\begin{lstlisting}
bool gr_exists(string id)
\end{lstlisting}
If this is the case one can remove it by
\begin{lstlisting}
void rem_gr(string id)
\end{lstlisting}
In order to replace a grid region by one with the same name but different attributes one calls one of the replace functions
\begin{lstlisting}
void replace_gr_equi(...)
void replace_gr_tan(...)
void replace_gr_log(...)
\end{lstlisting}
where the arguments are exactly the same as the one for the add functions for the corresponding type of grid (see tables \ref{tab:add_gr_equi}, \ref{tab:add_gr_tan} and \ref{tab:add_gr_log}). The only difference is that for the replace functions, you definitely have to give the name of the grid region which is to be replaced, as the \texttt{id} argument.

\subsection{Examples}\label{subsec:grid_region_examples}
\index{Grid region!add}
\index{Grid region!remove}
\index{Grid region!replace}

In this section we will show in some examples how the above grid regions can be added, replaced or removed from the multigrid. 

\begin{lstlisting}[caption={Example for the adding of equidistant grid regions},label={lst:add_gr_equi}]
multigrid mgrid;
mgrid.add_gr_equi(100, -1, 1, 0);
mgrid.add_gr_tan(100, 0.2, 0.5, 0.3, 0.01);
mgrid.add_gr_log(100, 100, 0.4, 0.7, 0.6, 1E-6, "lgrid");
mgrid.create()
\end{lstlisting}

In listing \ref{lst:add_gr_equi} we show an example for the adding of grid regions. At first an equidistant grid with $100$ points from $-4$ to $4$ is added to the multigrid. This defines the basic grid region and therefore the boundaries of the multigrid. The center point of a basic grid region (here at $0$) is obsolete because there is no cutting. Afterwards a tangential grid region with $200$ points from $0.2$ to $0.5$ with a center point at $0.3$ is added. The parameter $c$ is $0.01$ here. Finally a third, logarithmic grid region is added. It overlaps with the tangential grid region and an appropriate cutting point will be calculated such that the grid resolution remains constant across the cutting point. The logarithmic grid region is named   ``lgrid'' and has 100 points in both of the exponetial grid regions. Its boundaries are from $0.4$ to $0.7$, its center point $\omega_k$ is at $0.6$ and the half width of the linear region $\omega_0$ is $10^{-6}$.





\section{Cutting procedure}\label{sec:cutting_procedure}
\index{Cutting procedure}
If two grid regions intersect, they have to be cut or one of the grid regions has to be erased. This is the content of this chapter. 

\section{Subgrid and inserting grid regions}\label{sec:subgrids_and_insert}
\index{Subgrid}
\index{Inverse of the multigrid}
If all grid regions have gone through the cutting procedure explained in \ref{sec:cutting_procedure}, we are left with a bunch of non-intersection grid regions. These grid regions are now ready to be inserted into the basic grid region. But there is one difficulty we have not mentioned so far which is the fact that one has to take care of the inverse. Since all grid regions consist of one of the simple grid classes from chapter \ref{chapter:simple_grids}, their inverse mapping is know.

\section{Inverse of the multigrid}\label{sec:inverse_of_the_multigrid}
\index{Inverse of the multigrid}




\section{Fundamental and special grid regions}\label{sec:fundamental_and_special_grid_regions}
\index{Fundamental grid regions}
\index{Special grid regions}